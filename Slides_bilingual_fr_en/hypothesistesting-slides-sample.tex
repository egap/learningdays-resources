% Options for packages loaded elsewhere
\PassOptionsToPackage{unicode}{hyperref}
\PassOptionsToPackage{hyphens}{url}
\PassOptionsToPackage{dvipsnames,svgnames,x11names}{xcolor}
%
\documentclass[
  ignorenonframetext,
]{beamer}
\usepackage{pgfpages}
\setbeamertemplate{caption}[numbered]
\setbeamertemplate{caption label separator}{: }
\setbeamercolor{caption name}{fg=normal text.fg}
\beamertemplatenavigationsymbolsempty
% Prevent slide breaks in the middle of a paragraph
\widowpenalties 1 10000
\raggedbottom
\setbeamertemplate{part page}{
  \centering
  \begin{beamercolorbox}[sep=16pt,center]{part title}
    \usebeamerfont{part title}\insertpart\par
  \end{beamercolorbox}
}
\setbeamertemplate{section page}{
  \centering
  \begin{beamercolorbox}[sep=12pt,center]{part title}
    \usebeamerfont{section title}\insertsection\par
  \end{beamercolorbox}
}
\setbeamertemplate{subsection page}{
  \centering
  \begin{beamercolorbox}[sep=8pt,center]{part title}
    \usebeamerfont{subsection title}\insertsubsection\par
  \end{beamercolorbox}
}
\AtBeginPart{
  \frame{\partpage}
}
\AtBeginSection{
  \ifbibliography
  \else
    \frame{\sectionpage}
  \fi
}
\AtBeginSubsection{
  \frame{\subsectionpage}
}
\usepackage{amsmath,amssymb}
\usepackage{lmodern}
\usepackage{iftex}
\ifPDFTeX
  \usepackage[T1]{fontenc}
  \usepackage[utf8]{inputenc}
  \usepackage{textcomp} % provide euro and other symbols
\else % if luatex or xetex
  \usepackage{unicode-math}
  \defaultfontfeatures{Scale=MatchLowercase}
  \defaultfontfeatures[\rmfamily]{Ligatures=TeX,Scale=1}
\fi
% Use upquote if available, for straight quotes in verbatim environments
\IfFileExists{upquote.sty}{\usepackage{upquote}}{}
\IfFileExists{microtype.sty}{% use microtype if available
  \usepackage[]{microtype}
  \UseMicrotypeSet[protrusion]{basicmath} % disable protrusion for tt fonts
}{}
\makeatletter
\@ifundefined{KOMAClassName}{% if non-KOMA class
  \IfFileExists{parskip.sty}{%
    \usepackage{parskip}
  }{% else
    \setlength{\parindent}{0pt}
    \setlength{\parskip}{6pt plus 2pt minus 1pt}}
}{% if KOMA class
  \KOMAoptions{parskip=half}}
\makeatother
\usepackage{xcolor}
\newif\ifbibliography
\setlength{\emergencystretch}{3em} % prevent overfull lines
\providecommand{\tightlist}{%
  \setlength{\itemsep}{0pt}\setlength{\parskip}{0pt}}
\setcounter{secnumdepth}{-\maxdimen} % remove section numbering
\newenvironment{cols}[1][]{}{}

\newenvironment{col}[1]{\begin{minipage}{#1}\ignorespaces}{%
\end{minipage}
\ifhmode\unskip\fi
\aftergroup\useignorespacesandallpars}

\def\useignorespacesandallpars#1\ignorespaces\fi{%
#1\fi\ignorespacesandallpars}

\makeatletter
\def\ignorespacesandallpars{%
  \@ifnextchar\par
    {\expandafter\ignorespacesandallpars\@gobble}%
    {}%
}
\makeatother
\setbeamertemplate{footline}{\begin{beamercolorbox}{section in head/foot}
\includegraphics[height=.5cm]{../Images/egap-logo.png} \hfill
\insertframenumber/\inserttotalframenumber \end{beamercolorbox}}
\usepackage{tikz}
\usepackage{tikz-cd}
\usepackage{textpos}
\usepackage{booktabs,multirow,makecell}

\ifLuaTeX
  \usepackage{selnolig}  % disable illegal ligatures
\fi
\IfFileExists{bookmark.sty}{\usepackage{bookmark}}{\usepackage{hyperref}}
\IfFileExists{xurl.sty}{\usepackage{xurl}}{} % add URL line breaks if available
\urlstyle{same} % disable monospaced font for URLs
\hypersetup{
  pdftitle={Hypothesis Testing: Summarizing Information about Causal Effects \textbar{} Tests d'hypothèses : Résumer les informations sur les effets causaux},
  pdfauthor={Fill In Your Name},
  colorlinks=true,
  linkcolor={Maroon},
  filecolor={Maroon},
  citecolor={Blue},
  urlcolor={Blue},
  pdfcreator={LaTeX via pandoc}}

\title{Hypothesis Testing: Summarizing Information about Causal Effects
\textbar{} Tests d'hypothèses : Résumer les informations sur les effets
causaux}
\author{Fill In Your Name}
\date{21 March 2023}

\begin{document}
\frame{\titlepage}

\begin{frame}[allowframebreaks]
  \tableofcontents[hideallsubsections]
\end{frame}
\hypertarget{the-role-of-hypothesis-tests-in-causal-inference-le-ruxf4le-des-tests-dhypothuxe8se-dans-linfuxe9rence-causale}{%
\section{\texorpdfstring{The Role of Hypothesis Tests in Causal
Inference \textbar{} \emph{Le rôle des tests d'hypothèse dans
l'inférence
causale}}{The Role of Hypothesis Tests in Causal Inference \textbar{} Le rôle des tests d'hypothèse dans l'inférence causale}}\label{the-role-of-hypothesis-tests-in-causal-inference-le-ruxf4le-des-tests-dhypothuxe8se-dans-linfuxe9rence-causale}}

\begin{frame}{Key points for this lecture \textbar{} \emph{Points clés
de cette conférence}}
\protect\hypertarget{key-points-for-this-lecture-points-cluxe9s-de-cette-confuxe9rence}{}
\begin{cols}

\begin{col}{0.48\textwidth}

\footnotesize

\begin{itemize}
\item
  Statistical inference (e.g., hypothesis tests and confidence
  intervals) requires \textbf{inference} --- reasoning about the
  unobserved.
\item
  \(p\)-values require probability distributions.
\item
  Randomization (or Design) + a Hypothesis + a Test Statistic Function
  \(\rightarrow\) probability distributions representing the hypothesis
  (reference distributions)
\item
  Observed Values of Test Statistics + Reference Distribution
  \(\rightarrow\) \(p\)-value.
\end{itemize}

\end{col}

\begin{col}{0.04\textwidth}
~

\end{col}

\begin{col}{0.48\textwidth}

\footnotesize

\begin{itemize}
\item
  L'inférence statistique (par exemple, les tests d'hypothèse et les
  intervalles de confiance) nécessite une \textbf{inférence},
  c'est-à-dire un raisonnement sur des éléments non observés.
\item
  Les valeurs \(p\) nécessitent des distributions de probabilité.
\item
  Randomisation (ou plan) + une hypothèse + une fonction de statistique
  de test \(\rightarrow\) distributions de probabilité représentant
  l'hypothèse (distributions de référence)
\item
  Valeurs observées des statistiques du test + Distribution de référence
  \(\rightarrow\) \(p\)-value.
\end{itemize}

\end{col}

\end{cols}
\end{frame}

\begin{frame}{The role of hypothesis tests in causal inference I
\textbar{} \emph{Le rôle des tests d'hypothèse dans l'inférence causale
I}}
\protect\hypertarget{the-role-of-hypothesis-tests-in-causal-inference-i-le-ruxf4le-des-tests-dhypothuxe8se-dans-linfuxe9rence-causale-i}{}
\begin{cols}

\begin{col}{0.48\textwidth}

\footnotesize

\begin{itemize}
\item
  The \textbf{fundamental problem of causal inference} says that we can
  see only one potential outcome for any given unit.
\item
  So, if a counterfactual causal effect of the treatment, \(T\), for
  Jake occurs when \(y_{\text{Jake},T=1} \ne y_{\text{Jake},T=0}\), then
  how can we learn about the causal effect?
\item
  One solution is the \textbf{\href{estimation-slides.Rmd}{estimation}
  of averages of causal effects} (the ATE, ITT, LATE).
\item
  This is what we call Neyman's approach.
\end{itemize}

\end{col}

\begin{col}{0.04\textwidth}
~

\end{col}

\begin{col}{0.48\textwidth}

\footnotesize

\begin{itemize}
\item
  Le \textbf{problème fondamental de l'inférence causale} dit que nous
  ne pouvons voir qu'un seul résultat potentiel pour une unité donnée.
\item
  Ainsi, si un effet causal contrefactuel du traitement, \(T\), pour
  Jake se produit lorsque \(y_{text{Jake},T=1} \ne y_{text{Jake},T=0}\),
  comment pouvons-nous en savoir plus sur l'effet causal ?
\item
  Une solution consiste à \textbf{\href{estimation-slides.Rmd}{estimer}
  des moyennes des effets causaux} (ATE, ITT, LATE).
\item
  C'est ce que nous appelons l'approche de Neyman.
\end{itemize}

\end{col}

\end{cols}
\end{frame}

\begin{frame}{The role of hypothesis tests in causal inference II
\textbar{} \emph{Le rôle des tests d'hypothèse dans l'inférence causale
II}}
\protect\hypertarget{the-role-of-hypothesis-tests-in-causal-inference-ii-le-ruxf4le-des-tests-dhypothuxe8se-dans-linfuxe9rence-causale-ii}{}
\begin{cols}

\begin{col}{0.48\textwidth}

\footnotesize

\begin{itemize}
\item
  Another solution is to make \textbf{claims} or \textbf{guesses} about
  the causal effects.
\item
  We could say, ``I think that the effect on Jake is 5.'' or ``This
  experiment had no effect on anyone.'' And then we ask ``How much
  evidence does this experiment have about that claim?''
\item
  This evidence is summarized in a \(p\)-value.
\item
  We call this Fisher's approach.
\end{itemize}

\end{col}

\begin{col}{0.04\textwidth}
~

\end{col}

\begin{col}{0.48\textwidth}

\footnotesize

\begin{itemize}
\item
  Une autre solution consiste à faire des \textbf{revendications} ou des
  \textbf{conjectures} sur les effets causaux.
\item
  Nous pourrions dire : ``Je pense que l'effet sur Jake est de 5'' ou
  ``Cette expérience n'a eu d'effet sur personne''. Puis nous demandons
  : ``Quelle preuve cette expérience apporte-t-elle à cette affirmation
  ?''
\item
  Cette preuve est résumée dans une valeur \(p\).
\item
  C'est ce que nous appelons l'approche de Fisher.
\end{itemize}

\end{col}

\end{cols}
\end{frame}

\begin{frame}{The role of hypothesis tests in causal inference III
\textbar{} \emph{Le rôle des tests d'hypothèse dans l'inférence causale
III}}
\protect\hypertarget{the-role-of-hypothesis-tests-in-causal-inference-iii-le-ruxf4le-des-tests-dhypothuxe8se-dans-linfuxe9rence-causale-iii}{}
\begin{cols}

\begin{col}{0.48\textwidth}

\tiny

\begin{itemize}
\item
  The hypothesis testing approach to causal inference doesn't provide a
  best guess but instead tells you \emph{how much evidence or
  information the research design provides about a causal claim}.
\item
  The estimation approach provides a best guess but doesn't tell you how
  much you know about that guess.

  \begin{itemize}
  \tightlist
  \item
    \tiny For example, a best guess with \(N=10\) seems to tell us less
    about the effect than \(N=1000\).
  \item
    For example, a best guess when 95\% of \(Y=1\) and 5\% of \(Y=0\)
    seems to tell us less than when outcomes are evenly split betwee 0
    and 1.
  \end{itemize}
\item
  We nearly always report both since both help us make decisions: ``Our
  best guess of the treatment effect was 5, and we could reject the idea
  that the effect was 0 (\(p\)=.01).''
\end{itemize}

\end{col}

\begin{col}{0.04\textwidth}
~

\end{col}

\begin{col}{0.48\textwidth}

\tiny

\begin{itemize}
\item
  L'approche du test d'hypothèse pour l'inférence causale ne fournit pas
  une meilleure estimation, mais vous indique \emph{la quantité de
  preuves ou d'informations que le modèle de recherche fournit à propos
  d'une allégation causale}.
\item
  L'approche de l'estimation permet d'obtenir la meilleure estimation
  possible, mais ne permet pas de savoir ce que l'on sait de cette
  estimation.

  \begin{itemize}
  \tightlist
  \item
    \tiny Par exemple, une meilleure estimation avec \(N=10\) semble
    nous en dire moins sur l'effet que \(N=1000\).
  \item
    Par exemple, une meilleure estimation lorsque 95\% de \(Y=1\) et 5\%
    de \(Y=0\) semble nous en dire moins que lorsque les résultats sont
    également répartis entre 0 et 1.
  \end{itemize}
\item
  Nous indiquons presque toujours les deux, car ils nous aident à
  prendre des décisions : ``Notre meilleure estimation de l'effet du
  traitement était de 5, et nous pouvions rejeter l'idée que l'effet
  était de 0 (\(p\)=.01)''.
\end{itemize}

\end{col}

\end{cols}
\end{frame}

\end{document}
